\chapter{Methodik}
\label{ch:methodik}

\section{Akronyme}

\begin{description}[font=\sffamily\bfseries, leftmargin=0cm, itemsep=-0.15cm, style=nextline]
	\item \textbf{XML} die Extensible Markup Language
	\item \textbf{CDISC} Clinical Data Interchange Standards Consortium
	\item \textbf{CDISC ODM} von CDISC entwickeltes XML-Format \enquote{Operational Data Model}
	\item \textbf{lsa} Dateiendung des LimeSurvey Archives (siehe \cref{lsa})
	\item \textbf{lsr} Dateiendung der LimeSurvey Response Datei (siehe \cref{lsa})
	\item \textbf{lss} Dateiendung der LimeSurvey Struktur Datei (siehe \cref{lsa})
	\item \textbf{EDC} Electronic Data Capture, das Sammeln und Verarbeiten von Daten
	\item \textbf{DOM} Document Object Model. Bietet die Möglichkeit, die Hierarchie der XML-Knoten in Baumform darzustellen und so zu navigieren/ den Baum zu bearbeiten
	\item \textbf{SAX} Simple API for XML. Ein Standard, welcher beschreibt, wie man ein XML-Dokument parsen kann. Dieses wird sequentiell eingelesen und für definierte Ereignisse wird eine vorgegebene Rückruf-Funktion aufgerufen. Ein Programm kann eigene Funktionen registrieren und so das Dokument verarbeiten.
\end{description}

\section{LimeSurvey}

- Umfragewerkzeug, sammeln von Meinungen/Interessen/Entscheidungsgrundlagen
- Firma gleichen Namens aus D
- Wird von den Entwicklern in der Cloud-Edition angeboten
- 4 Pläne
	- Free
	- Basic (34 Euro/Monat)
	- Expert (29Euro/Monat)
	- Enterprise (74 Euro/Monat)
- Code auch Open-Source

\subsection{Fragegruppen}

- Name/Beschreibung/Randomisierungsgruppe/Relevanz
- Kann beliebig viele Fragen enthalten

\subsection{Fragetypen}

- 36 Fragetypen (manche sind keine "Fragen" per se, davon gibt es 28), 5 Gruppen
- Bedingungen

- Single Choice

- Skala 1-5
- Bootstrap/Dropdown/Radio sind untersch. Darstellungen
- Liste mit Kommentar hat nur Kommentar
- Image Select List Radio hat Bild dazu

- Arrays

- Man kann aus einer Festen Liste an Möglichkeiten eine pro Subfrage Auswählen
- 5 Punkte, 10 Punkte, Erhöhung/Gleich/Erniedrigung, Ja/Nein/Unsicher, Selbst definierte Möglichkeiten
- Nach Spalte: Gleiches wie Freitext, nur Achsen getauscht
- Dual Scale: Zwei mal selbst definierte Möglichkeiten, aus beiden Skalen eine Antwort pro Subfrage auswählbar.
- Matrizen
- Freitext oder freie Zahl

- Multiple Choice

- Darstellungen: Bootstrap, simples Multiple Choice
- Mit Bild oder mit Kommentar (Ein Kommentar pro Antwortfeld)

- Textfragen

- Browser Detect
- Kurz/Normal/Lang-er Freitext
- Mehrere kurze Freitexte
- Input on Demand

- Maskenfragen

- Tag/Zeit
- Ja/Nein
- Gleichung
- Dateiupload
- Geschlecht
- Sprachumschaltung
- Zahleneingabe (Auch mehrfach, mit Limitierbaren Maximal/Minimal-Werten/Summen, Genauen Gesamtergebnissen und nur Ganzzahl)
- Ranking (Advanced)
- Textanzeige

\subsection{Export}
\subsubsection{LimeSurvey Archiv}
\label{lsa}
Das LimeSurvey Archiv ist eine von sieben Möglichkeiten, Daten von einer LimeSurvey Umfrage zu exportieren.
Ein solches Archiv ist eine komprimierte Datei im .lsa-Format, welche mehrere extrahierbare Dateien enthält.
Die Zahl und Art der Dateien ist dabei abhängig von den Einstellungen. Zwei Dateien sind immer enthalten:\\
Die erste Datei enthält die Umfrage-Struktur sowie Informationen über die Art und Weise, wie die Fragen darstellt werden sollen (die .lss-Datei), die zweite Datei enthält die Antworten der Teilnehmer (.lsr-Datei).
Ausdrücklich erwähnt wird dabei, dass Dateien, die als Antwort auf eine Frage hochgeladen wurden, nicht Teil des Archivs sind.
Weitere optionale Dateien sind eine Token-Datei, welche Daten über die % TODO

\subsubsection{Weitere Exportmöglichkeiten}
\begin{itemize}
	\item[LSS] Es ist möglich, nur die im LSA enthaltene LSS-Datei zu exportieren, das wird mittels dieser Option gemacht.
	\item[Excel/.csv] Hier sind weitere Einstellungen möglich, wie das Exportieren eines Teils der Antworten oder die Wahl eines bestimmten Formates (Word, Excel, CSV, HTML, PDF).
	\item[SPSS] SPSS ist ein Software-Paket, welches zur statistischen Analyse von Daten genutzt wird. Auch hier kann ausgewählt werden, welche Antworten exportiert werden sollen. Die Nutzung der Open-Source Version PSPP ist auch möglich
	\item[R] R ist eine Alternative zu SPSS, hier werden allerdings alle Daten exportiert
	\item[STATA-xml] Auch STATA ist eine komerzielle Lösung für Datenanalyse wie SPSS. Hierfür werden die Daten von LimeSurvey direkt in das proprietäre STATA-Format umgewandelt.
	\item[VV] Durch "vertical verfication" ist es möglich, die Antworten zu modifizieren und die modifizierte Datei dann wieder zu importieren
\end{itemize}

\section{ODM}
- Ein Datenstandard im XML-Format
- entwickelt von der CDISC
- Ziel: Ein klares, einheitliches Format für Klinische Daten
- Zusätzliche Daten wie Metadaten oder admin. Daten werden auch gespeichert
- Plattform-unabhängig
- Unabhängig von spezifischen Firmen
- Mittlerweile viel genutzt in EDC Tools

\section{dom4j}

- API für den Zugriff und die Verarbeitung von XML-Dokumenten
- Bietet simple Wege, um existierende Funktionen wie XPath oder Parsing mittels DOM/SAX zu nutzen
- Auch Dokumente erstellen möglich
%%
%\section{Mathematische Notation}
%\label{s:notation}

%Mathematische Formeln können mittels der \verb+\begin{align}...\end{align}+ Umgebung gesetzt werden:

%\begin{align}
%f(n) & =
%	\begin{cases}
%		n/2, & \text{wenn }n\text{ gerade,}\\
%		3n+1, & \text{wenn }n\text{ ungerade.}
%	\end{cases}
%\label{eq:f} \\
%%
%g(n) & = \frac{n}{2} \label{eq:g}
%\end{align}
%
%\section{Algorithmus}
%\label{s:algorithmus}
%
%Eigene Algorithmen beschreibt man am Besten mit Hilfe von Pseudo-Code und dem Paket \verb+algorithm+.
%
%\begin{algorithm}
%\caption{Algorithmus}
%\label{alg:alg}
%\begin{algorithmic}
%\algsetup{indent=2em}
%
%\REQUIRE Argument $n\in\mathbb{N}$
%\STATE $a = 0$
%\FOR{ $i=0,\dots,n$}
%	\STATE $a = a + 1$
%\ENDFOR
%\RETURN $a$
%\end{algorithmic}
%\end{algorithm}
