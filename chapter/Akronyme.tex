\chapter{Akronyme}

\begin{description}[font=\sffamily\bfseries, leftmargin=0cm, itemsep=-0.15cm, style=nextline]
	\item \textbf{EDC} \el{Electronic Data Capture} Das Sammeln und Verarbeiten von Daten
	\item \textbf{RegEx} \el{Regular Expression} Ein Ausdruck, der genutzt werden kann, um Zeichenketten auf eine bestimmte Struktur zu überprüfen
	\item \textbf{CSS} \el{Cascading Style Sheet} Eine Stylesheet-Sprache, welche die Gestaltung von HTML- oder XML-Elementen bestimmt
	\item \textbf{IMI} \el{Institut für Medizinische Informatik} Eines der Institute der WWU. Diese Arbeit ist in Kooperation mit ihnen entstanden
	\item \textbf{XML} \el{eXtensible Markup Language} Eine Auszeichnungssprache zum Speichern hierarchisch strukturierter Daten 
	\item \textbf{XSD} \el{XML Schema Definition} Eine Datei, welche beschreibt, wie ein XML-Dokument aufgebaut sein sollte, um einer bestimmten Definition zu entsprechen.
	\item \textbf{SAX} \el{Simple API for XML} Ein Standard, welcher beschreibt, wie man ein XML-Dokument parsen kann. Dieses wird sequentiell eingelesen und für definierte Ereignisse wird eine vorgegebene Rückruf-Funktion aufgerufen. Ein Programm kann eigene Funktionen registrieren und so das Dokument verarbeiten.
	\item \textbf{DOM} \el{Document Object Model} Bietet die Möglichkeit, die Hierarchie der XML-Knoten in Baumform darzustellen und so zu navigieren/ den Baum zu bearbeiten
	\item \textbf{CDISC} \el{Clinical Data Interchange Standards Consortium} Eine Non-Profit-Organisation, welche Standards zum Austausch von Daten aus klinischen Studien entwickelt
	\item \textbf{CDISC ODM} \el{Operational Data Model} von CDISC entwickeltes XML-Format (siehe \cref{m:odm})
	\item \textbf{lsa} Abkürzung für und Dateiendung des LimeSurvey Archives (siehe \cref{m:lsa})
	\item \textbf{lsr} Abkürzung für und Dateiendung der LimeSurvey Response Datei (siehe \cref{m:lsa})
	\item \textbf{lss} Abkürzung für und Dateiendung der LimeSurvey Struktur Datei (siehe \cref{m:lsa})
\end{description}
