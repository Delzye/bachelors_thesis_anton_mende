\chapter{Fazit}
\label{ch:fazit}

% Dieses Kapitel bildet die abschließende Zusammenfassung der Arbeit. Dazu können die folgende Punkte behandelt werden:
% \begin{itemize}
% 	\item Reflexion: wurden die Ziele der Arbeit erreicht?
% 	\item mögliche Erweiterungen und Verbesserungen (\glqq future work\grqq)
% \end{itemize}

Eine Umwandlung von LimeSurvey-Archiven in das Operational Data Model ist grundsätzlich mit kleineren Einschränkungen möglich.
Das in dieser Arbeit erstellte Mapping wurde implementiert und seine Praxistauglichkeit damit bewiesen.

Das Arbeiten mit dem LSA-Format, was außerhalb von LimeSurvey durch mangelnde Dokumentation früher anstrengend gewesen sein muss, wurde nun erleichtert.
Einerseits kann jeder nun sehr schnell ein Verständnis für den Format-Aufbau erlangen, da selbiger in dieser Arbeit präzise analysiert wurde, andererseits kann jeder den implementierten Konverter nutzen, um seine Daten direkt im ODM-Format vorliegen zu haben.

Da ODM eigentlich dazu gedacht ist, wesentlich komplexere Studien-Strukturen darzustellen, bleiben einige Features hier ungenutzt, 
gleichzeitig besitzt LimeSurvey erheblich mehr Möglichkeiten, die visuelle Darstellung und andere Eigenschaften einer Frage zu beeinflussen.
Das macht es notwendig, vieles entweder zu simplifizieren oder wegzulassen, wenn man eine Konvertierung vornehmen möchte.

Dennoch ist es in dieser Arbeit gelungen, alle relevanten Metadaten und klinischen Daten aus LimeSurvey zu extrahieren und ohne einen größeren Verlust an Informationen in ODM einzupflegen.
Das hier erstellte Formular kann nun entweder für sich genutzt werden oder in anderen ODM-Dateien mit größeren Studien eingebunden werden.

Dadurch wird die Interoperabilität erheblich erhöht und der Austausch von Daten zwischen verschiedenen Systemen stark simplifiziert.
Dieser Konverter reiht sich in eine längere Reihe an Konvertern ein, welche andere Formate zum Speichern medizinischer und klinischer Daten entweder zu oder von ODM umwandeln.
Durch dieses sich ständig erweiternde Netzwerk wird es in Zukunft immer einfacher, ein beliebiges Datenformat in ein anderes umzuwandeln, indem man einen Mittelweg über ODM geht.
Auch wird das Speichern der Daten immer einfacher, wenn alles in einem einzigen Format gespeichert werden kann.
Dafür ist unter anderem auch die Qualität des Konverters entscheidend, da der Datenverlust beim Konvertieren minimal sein soll.

Die bisherige Implementierung macht hier bereits einen guten Job, allerdings kann man, mit genügend Zeit und wenn man die Notwendigkeit sieht, noch weitere Features hinzufügen:
\begin{itemize}
	\item Aufgaben in der Richtung LSA $\rightarrow$ ODM
	\begin{itemize}
		\item Mehrere Umfragen in einer LSS-Datei unterstützen
		\item Fragen in verschiedenen Sprachen in einem ItemDef Element unterbringen (Bisher werden Fragen doppelt übernommen)
		\item Unterstützung für mehr Formate bei \el{Datum/Zeit-Fragen}
		\item Übernahme von regulären Ausdrücken zur Validierung der Antwort-Struktur als \el{RangeCheck}
		\item Übernehmen, wie viele Ziffern eine Zahl bei einer Antwort lang sein darf
	\end{itemize}
	\item Aufgaben in der Richtung ODM $\rightarrow$ LSA
	\begin{itemize}
		\item Konvertierung der Antworten in einer LSR-Datei
		\item Packen beider Dateien in ein Archiv
		\item Eine Möglichkeit einfügen, alle Formulare zu konvertieren
		\item Einstellbare Werte beim Schreiben der LSS-Datei (Sprache, Datenbank-Version)
		\item Fragen für weitere Datentypen wie Datum/Zeit einfügen
	\end{itemize}
\end{itemize}
