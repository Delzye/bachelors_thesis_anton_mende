\chapter{Einleitung}
\label{ch:einleitung}

Das Thema dieser Arbeit ist die Konvertierung von LimeSurvey Archiven (LSA) in das Operational Data Model (ODM).
LimeSurvey ist ein Werkzeug, mit dem man einfach Umfragen erstellen kann, welche dann wiederum von beliebig vielen Teilnehmern beantwortet werden können.
Umfragewerkzeuge lassen sich sehr vielseitig einsetzen, um die Meinungen von Kunden, Patienten oder jedem anderen Menschen einzuholen.
Gerade in den heutigen Zeiten, wo immer mehr Daten gesammelt und verarbeitet werden können, ist es wichtig, Kompatibilität zwischen verschiedenen Werkzeugen und Formaten herstellen zu können.
Das Institut für Medizinische Informatik an der WWU ist eine der vielen Institutionen, welche LimeSurvey nutzt, um die Meinungen und Erfahrungen von Patienten, das \enquote{electronic partient-reported outcome}(ePRO), einzuholen.

Auch das Bundesministerium für Bildung und Forschung (BMBF) hat schon vor Jahren erkannt, dass die digitale Verwaltung medizinischer Datensätze in der Medizininformatik stetig an Relevanz gewinnt und 150 Millionen Euro für eine Initiative bereitgestellt, welche verschiedene Institutionen in der Aufbau- und Vernetzungsphase unterstützen soll \cite{bmbf_medinfo}.

\section{Problem}
Beide Dateiformate sind in der \enquote{eXtensible Markup Language}(XML) geschrieben.
Der XML-Standard lässt dem Programmierer sehr viel Freiheit, was das Design eines XML-Formates angeht. Dadurch sind verschiedene XML-Formate im Regelfall nicht kompatibel beziehungsweise untereinander austauschbar.
Das macht es notwendig, XML-Dokumente zu konvertieren, um ein Dokument in einem Format mit einem Werkzeug nutzen zu können, welches nur ein anderes XML-Format unterstützt.
Als Zielformat wird ODM-XML des \textit{Clinical Data Interchange Standards Consortium} (CDISC) verwendet.
ODM wurde mit dem Ziel entwickelt, den Austausch und die Archivierung von Forschungsdaten und anderen damit verbundenen Daten zu ermöglichen.
Durch die Unabhängigkeit des Formates von spezifischen Plattformen oder Firmen wird es durch weit mehr Werkzeuge unterstützt als ein Format wie LSA, welches von LimeSurvey selbst und zu deren eigenen Zwecken entwickelt wurde.
Dies sieht man auch an dem Institut für Medizinische Informatik, welche dieses Projekt betreut.
Dort wird ODM zur Verwaltung klinischer Daten genutzt, viele davon sind über das \enquote{Medical Data Models}-Portal (MDM-Portal)\cite{mdm} abrufbar.
Gleichzeitig ist aber auch LimeSurvey das Umfragewerkzeug der Wahl, um Daten von Patienten zu sammeln.
Diese gesammelten Daten müssen dann exportiert und konvertiert werden, um sie in bestehenden Systemen einpflegen zu können.

\section{Ziel}
Ziel der Arbeit ist es, genau darzustellen, wie das LimeSurvey Archiv und die darin enthaltenen Dateien aufgebaut sind.
Dann soll gezeigt werden, wie man ein LimeSurvey Archiv (LSA) in ein ODM Dokument umwandeln kann, indem ein Mapping erstellt wird.
Anschließend soll anhand des Mappings ein Konverter implementiert werden, welcher die Umwandlung durchführt.
Es sollen so viele der Forschungsdaten wie möglich übertragen werden, zusätzliche Daten, welche z.B. die Darstellung der Daten betreffen, sollen nicht mit übertragen werden.
Auch ein Konverter in die andere Richtung, also von ODM Dateien zu LimeSurvey, soll diskutiert werden und eine möglichst funktionsreiche Version soll anschließend implementiert werden.
