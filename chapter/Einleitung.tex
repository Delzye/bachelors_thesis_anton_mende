\chapter{Einleitung}
\label{ch:einleitung}

Das Thema dieser Arbeit ist die Konvertierung von LimeSurvey Archiven in das Operational Data Model (ODM).
LimeSurvey ist ein Werkzeug, mit dem man einfach Umfragen erstellen kann, welche dann wiederum von beliebig vielen Teilnehmern beantwortet werden können.
Umfragewerkzeuge lassen sich sehr vielseitig einsetzen, um die Meinungen von Kunden, Patienten oder jedem anderen Menschen einzuholen.
Gerade in den heutigen Zeiten, wo immer mehr Daten gesammelt und verarbeitet werden können, ist es wichtig, Kompatibilität zwischen verschiedenen Werkzeugen und Formaten herstellen zu können.
Das Institut für Medizinische Informatik an der WWU ist eine der vielen Institutionen, welche LimeSurvey nutzt, um die Meinungen und Erfahrungen von Patienten einzuholen.

\section{Problem}
Der XML-Standard lässt dem Programmierer sehr viel Freiheit, was das Design eines XML-Formates angeht. Dadurch sind verschiedene XML-Formate im Regelfall nicht kompatibel beziehungsweise untereinander austauschbar.
Das macht es notwendig, XML-Dokumente zu konvertieren, um ein Dokument in einem Format mit einem Werkzeug nutzen zu können, welches nur ein anderes XML-Format unterstützt.
Als Zielformat wird ODM-XML des \textit{Clinical Data Interchange Standards Consortium} (CDISC) verwendet.
ODM wurde mit dem Ziel entwickelt, den Austausch und die Archivierung von Forschungsdaten und anderen damit verbundenen Daten zu ermöglichen.
Durch die Unabhängigkeit des Formates von spezifischen Plattformen oder Firmen wird es durch weit mehr Werkzeuge unterstützt als ein Format wie LSA, welches von LimeSurvey selbst und zu deren eigenen Zwecken entwickelt wurde.
Dies sieht man auch an dem Institut für Medizinische Informatik, welche dieses Projekt betreuen.
Dort wird ODM täglich zur Verwaltung klinischer Daten genutzt, gleichzeitig ist aber auch LimeSurvey das Umfragewerkzeug der Wahl, um Daten von Patienten zu sammeln.
Diese gesammelten Daten müssen dann exportiert und konvertiert werden, um sie in bestehenden Systemen einpflegen zu können.

\section{Ziel}
Ziel der Arbeit ist es, genau darzustellen, wie man ein LimeSurvey Archiv (LSA) in ein ODM Dokument umwandeln kann und einen Konverter zu implementieren, welcher das bewerkstelligt.
Es sollen so viele der Forschungsdaten wie möglich übertragen werden, zusätzliche Daten, welche z.B. die Darstellung der Daten betreffen, sollen nicht mit übertragen werden.

% Im Einleitungskapitel werden dem Leser das Thema, die Grundlagen und vergleichbare Arbeiten präsentiert.

% Im Folgenden werden ein paar wichtige Tipps zum Erstellen der wissenschaftlichen Arbeit festgehalten. Die vorliegende inhaltliche Strukturierung dient nur zur Orientierung und ist nicht verbindlich. Insbesondere bei Seminararbeiten kann die Struktur abweichen. Beispielsweise ist dort nicht generell eine Implementierung gefordert (und folglich müssen auch keine Ergebnisse beschrieben werden). Auch Zusammenfassung, Danksagung, eidesstattliche Erklärung, Abbildungs- und Tabellenverzeichnis können bei Seminararbeiten weggelassen werden.


% \section{Referenzierung von Unterkapiteln}
% \label{s:ref} 
% Man sollte direkt jedes Kapitel, Unterkapitel und jede Formel mit einem Label versehen \verb+\label{}+ um eine konsistente Referenzierung im gesamten Dokument zu ermöglichen. Eine Referenzierung im Fließtext lässt sich mittels \verb+\ref{}+ umsetzen.


% \section{Zeilenumbrüche und Absätze}
% \label{s:zeilenumbruch} 
% Zeilenumbrüche sollten im Quelltext immer mittels einer Leerzeile umgesetzt werden, um eine automatische Texteinrückung in der darauffolgenden Zeile zu ermöglichen. Dies macht das Lesen der Arbeit einfacher. Auf die Verwendung des LaTeX-Kommandos \verb+\\+ sollte verzichtet werden.


% \section{Einfügen von Grafiken}
% \label{s:grafik} 
% Grafiken sollten mittels der \verb+\figure+-Umgebung eingebettet werden. Um die Druckqualität und Wiederverwendbarkeit der Grafiken für Vorträge, Poster, etc.\ zu erhöhen sind Vektorgrafiken (z.B.\ .eps oder .pdf) zu bevorzugen. Zur Erzeugung und Konvertierung von Vektorgrafiken ist die OpenSource Software \emph{Inkscape} zu empfehlen.

% Um mehrere Bilder horizontal anzuordnen, sollte die \verb+\subfigure+-Umgebung verwendet werden. Diese Bilder können dann mit \verb+\subref+ oder \verb+\ref+ referenziert werden und erscheinen im Text als \subref{fig:subfig1} oder \ref{fig:subfig1}.

% \begin{figure}
% \begin{center}
% 	\subfigure[Das Logo]{
		% \includegraphics[width=0.45\columnwidth]{./img/wwu-logo-neu.pdf}
		% \label{fig:subfig1}
	% }
	% \subfigure[Noch einmal das Logo]{
		% \includegraphics[width=0.45\columnwidth]{./img/wwu-logo-neu.pdf}
		% \label{fig:subfig2}
	% }
% \end{center}
% \caption{Das Logo der WWU}
% \label{fig:histogram}
% \end{figure}


% \section{Wissenschaftliches Zitieren}
% \label{s:zitat}
% Für das Referenzieren von wissenschaftlicher Literatur wie Fachbüchern, Konferenzpapern und Veröffentlichungen in Wissenschaftsmagazinen gibt es unterschiedliche Konventionen. Je nach Fachrichtung weichen Layout und Zitationsstil sehr stark voneinander ab \cite{jele2010}. Wir empfehlen aus Gründen der Einheitlichkeit die Verwendung der \emph{Bibtex}-Umgebung. Die zitierte Literatur kann ausgelagert in einer Datei (z.B.: Quellen.bib) gepflegt werden und mittels des \verb+\cite+-Kommandos referenziert werden.


% \section{Zusammenfassung}
% \label{s:zusammenfassung}
% Zum Ende eines längeren Kapitels bietet es sich häufig an eine Zusammenfassung der wichtigsten Punkte zu liefern. Dies erleichtert das Lesen und den Übergang zum nächsten Kapitel.
