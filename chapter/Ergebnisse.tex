\chapter{Ergebnisse}
\label{ch:ergebnisse}

\section{Analyse des LSS- und LSR-Formates}

% - Keine Dokumentation zum Aufbau des XML vorhanden
% - Lösung: Testdokumente erstellen und von da aus Format herausfinden
% 	- Erst simples Dokument mit Textfragen und zwei Fragegruppen

In der Online-Dokumetation von LimeSurvey gibt es zwar eine Sektion zum Thema Export und Export-Formate, dieses Kapitel war aber bis vor kurzem leer.
Auch jetzt enthält das Kapitel nur eine oberflächliche Beschreibung der Formate, die genaue Struktur wird nicht erklärt.
Da diese allerdings benötigt wird, um eine Konvertierung vornehmen zu können, muss zunächst ermittelt werden, wie die .lss- und .lsr-Dateien genau aufgebaut sind.


Dies wurde bewerkstelligt, indem zunächst ein simples Dokument erstellt wurde, welches eine Fragegruppe und drei Freitextfragen enthält.
Dafür wurde eine Instanz der LimeSurvey Community Edition benötigt, diese aufzusetzen war dank existierenden Docker-Compose-Dateien relativ simpel.
Nachdem die Struktur der Umfrage selbst, sowie der Fragegruppen so ermittelt wurde, wurden komplexere Dateien erstellt.
Diese enthielten zunächst Fragen mit festen Antwortmöglichkeiten, also vor allem Maskenfragen und Multiple Choice Fragen.
Nachdem so auch die Struktur der Antworten deutlich geworden war, wurden zuletzt die Matrixfragen eingebaut und die Struktur der Subfragen ermittelt.
Das Ergebis dieser Analyse wird im Folgenden dargestellt (Es werden nicht alle existierenden Elemente angesprochen, sondern nur die für diese Arbeit relevanten, eine vollständige Auflistung kann im %TODO 
gefunden werden):

\subsection{Grundlegende Struktur}

Das Root-Element in LSS ist \el{document}.
Hier sind zuerst grundlegende Informationen wie die Datenbank-Version und den Typ des Dokuments enthalten.
Jedes der größeren Hauptelemente in \el{document} hat die gleiche Struktur.
Es gibt zwei Elemente, \el{fields} und \el{rows}. In \el{rows} gibt es \el{row} Elemente, welche die Informationen selbst in weiteren Elementen enthalten.
In \el{fields} gibt es \el{fieldname} Elemente, wobei es für jedes mögliche Element in \el{rows/row} ein \el{filedname} Element mit dem Namen als Text gibt.

\subsection{Fragegruppen}

Im Element \el{groups} werden Metadaten über Fragegruppen gesammelt, wie zum Beispiel die Gruppen-ID.
Diese sind aber auch in \el{group\_l10ns} enthalten, darüber hinaus sind auch noch weitere Inhalte wie Gruppenname und Sprache in diesem Element.
Daher werden später keine Informationen aus \el{groups} später verwendet.
Ein \el{row} Element steht hier für eine Fragegruppe.

\subsection{Fragen}

Das Element \el{questions} enthält Metadaten über Fragen, wie \el{qid}, \el{gid}, \el{title}, \el{type}.
Ein \el{row} Element steht hier für eine Frage, pro richtiger Frage in der Umfrage gibt es ein Element in \el{questions}.
\el{subquestions} hat Subfragen von Arrays, Matrizen et cetera. Diese sind via \el{parent\_qid} and ein Element aus \el{questions} gekoppelt.
Auch in \el{subquestions} sind nur Metadaten enthalten, jede Subfrage hat, wie die richtigen Fragen auch, eine Fragen-ID.
In \el{question\_l10ns} gibt es nun die tatsächliche Frage in \el{question}, weiterhin gibt es mit \el{help} einen Hilfstext für die Frage, \el{language} gibt die Sprache des Fragetextes an.
Für jedes Element aus \el{question} und \el{subquestion} gibt es hier ein Element pro Sprache.
Referenziert werden diese mit der \el{qid}.

\el{question\_attributes} enthält Informationen über eine Frage wie ein Prä-/Suffix zu der Antwort, RegEx-Validations-Ausdrücke für die Antwort, Timings und Informationen zur Darstellung einer Frage (Textfeldbreite, Default-Antworten).

\subsection{Antwortmöglichkeiten}

Es gibt eine Reihe an Fragen, für die es eine Menge an vordefinierten Antworten gibt.
Diese sind entweder schon durch die Frage festgelegt, wie bei dem Fragetyp \el{5 Punkte Wahl} oder dem Typ \el{Geschlecht}, oder der Umfrage-Ersteller kann sie selber angeben.
Sind die Möglichkeiten schon durch den Typ festgelegt, werden die Antwortmöglichkeiten implizit ermittelt und nie konkret im Dokument niedergeschrieben.
Hat der Umfrage-Ersteller die Möglichkeiten selber festgelegt, werden diese in \el{answers} und \el{answer\_l10ns} gespeichert.
In \el{answers} gibt es dabei wieder Metadaten wie \el{qid}, \el{aid}, und einen \el{code}. 
In \el{answer\_l10ns} hingegen gibt es den Antworttext in \el{answer}, eine \el{aid} zur Verknüpfung mit den Metadaten und die Sprache in \el{language}.
Für jedes Element aus \el{answers} gibt es pro Sprache ein Element in \el{answer\_l10ns}.

\subsection{Umfrage-Metadaten}

In \el{surveys} findet man Metadaten über die Umfrage, allerdings sind keine davon für die Konvertierung relevant.
Beispiele wären Daten darüber, ob Willkommenstexte angezeigt werden sollen oder ob IP-Adressen der Teilnehmer gespeichert werden sollen.
\el{surveyls\_languagesettings} enthält relevante Informationen wie die \el{surveyls\_id}, den Titel in \el{surveyls\_title} und eine Beschreibung der Umfrage in \el{surveyls\_description}.
\el{themes} und \el{themes\_inherited} enthalten Infos über die visuelle Darstellung der Umfrage in LimeSurvey.

\subsection{LSR-Aufbau}

Auch für die Antworten wurde die gleiche Strategie wie für die Umfragestruktur verwendet.
Die Ergebnisse sind wie folgt:

Das Hauptelement ist wieder \el{Document}, zuerst gibt es wieder einige Metadaten wie der Typ des Dokuments, die Datenbank-Version und die Sprache.
Dann gibt es ein \el{responses} Element, welches dieselbe grundlegende Struktur wie die Elemente in der LSS-Datei besitzt.
Für jeden Teilnehmer der Umfrage gibt es ein Element vom Typ \el{row}.
Dies enthält eine ID, das Absendedatum, die Sprache und einen Seed.
Dann kommen die Antworten, wobei es für jede Frage bzw. Subfrage ein Element mit folgendem Namensschema gibt: \el{\_\{Survey-ID\}X\{GID\}X\{QID\}\{SQID\}?\{ext\}?}, wobei \el{ext} folgendes sein kann (\enquote{other}$\vert$\enquote{comment}).
Man kann bereits am Format sehen, dass es auch für die \enquote{other} und Kommentar-Felder hier eine separate Antwort gibt.

\subsection{Erstellung eines XML-Schemas}

\section{Mapping}

\subsection{Dummy Elemente in ODM}

- ODM kann mehr Beziehungen in Studien darstellen
- Es müssen Dummy-Elemente angelegt werden, weil LimeSurvey diese Elemente nicht hat, sie aber in ODM vorkommen müssen
- Betroffen: GlobalVariables, StudyEvent

\subsection{Umfrage-Eigenschaften}

- Eine Umfrage wird als ein \el{Form} dargestellt
- Titel, Beschreibung und ID der Umfrage werden zu Titel, Beschreibung und ID des \el{Forms}

\subsection{Fragegruppen}

- Aus jeder LS Gruppe wird eine ItemGroup in ODM
- Eigenschaften werden 1:1 abgebildet

\subsection{Fragen}

- Aus einer Frage in LS werden pot. mehrere ItemDef's in ODM (fragetypabhängig)
- Elemente aus \el{Questions} und \el{Subquestions} und \el{Question\_l10ns} werden genutzt

\subsection{Themes und Frageattribute}

- Werden nicht übernommen
	- ODM dient zum Austausch der Fragen/Antworten, nicht die Art und Weise, wie sie dargestellt werden

\section{Implementierung}

\subsection{Java}

- Bietet sehr mächtige Werkzeuge zur Bearbeitung von XML
	- Kann z.B. mit mehreren Attributen in einem Element umgehen
	- Unterstützt XSD-1.1
- Wird beim IMI bereits viel genutzt

\subsection{Eingabe}

- Archiv
	- muss entpackt werden

\subsection{XSD-Validierung}

- Nutze Xerces zusammen mit der erstellten XSD um eine Eingabedatei zu prüfen
- Prüfung ist nicht bindend
	- Auch invalide Datei wird weiter verarbeitet
	- Grund: Starke Versionsabhängigkeit
		 - Invalide lss-Datei kann eventuell trotzdem erfolgreich umgewandelt werden
	- Zur Information für Anwender: Potentiell problematisch/inkompatibel mit Konverter

\subsection{Parsing der LimeSurvey Struktur}

\subsection{Parsing der LimeSurvey Antworten}

\subsection{Ausgabe als ODM-Datei}

% Dieses Kapitel sollte die Ergebnisse beinhalten, die mit den Methoden aus \autoref{ch:methodik} erstellt wurden.

% \begin{table}[h]
% \caption{Beispieltabelle}
% \begin{center}
% 	\begin{tabular}{|c||c|c|}
% 		\hline
% 		Spalte1 & Spalte2 & Spalte3 \\ 
% 		\hline\hline
% 		   1    &    2    &    3    \\ 
% 		\hline
% 	\end{tabular}
% \end{center}
% \label{tbl:table}
% \end{table}
