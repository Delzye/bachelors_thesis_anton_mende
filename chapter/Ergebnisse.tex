\chapter{Ergebnisse}
\label{ch:ergebnisse}

\section{Mapping}

\subsection{Dummy Elemente in ODM}

- ODM kann mehr Beziehungen in Studien darstellen
- Es müssen Dummy-Elemente angelegt werden, weil LimeSurvey diese Elemente nicht hat, sie aber in ODM vorkommen müssen
- Betroffen: GlobalVariables, StudyEvent

\subsection{Umfrage-Eigenschaften}

- Eine Umfrage wird als ein \el{Form} dargestellt
- Titel, Beschreibung und ID der Umfrage werden zu Titel, Beschreibung und ID des \el{Forms}

\subsection{Fragegruppen}

- Aus jeder LS Gruppe wird eine ItemGroup in ODM
- Eigenschaften werden 1:1 abgebildet

\subsection{Fragen}

- Aus einer Frage in LS werden pot. mehrere ItemDef's in ODM (fragetypabhängig)
- Elemente aus \el{Questions} und \el{Subquestions} und \el{Question\_l10ns} werden genutzt

\section{Implementierung}

% Dieses Kapitel sollte die Ergebnisse beinhalten, die mit den Methoden aus \autoref{ch:methodik} erstellt wurden.

% \begin{table}[h]
% \caption{Beispieltabelle}
% \begin{center}
% 	\begin{tabular}{|c||c|c|}
% 		\hline
% 		Spalte1 & Spalte2 & Spalte3 \\ 
% 		\hline\hline
% 		   1    &    2    &    3    \\ 
% 		\hline
% 	\end{tabular}
% \end{center}
% \label{tbl:table}
% \end{table}
