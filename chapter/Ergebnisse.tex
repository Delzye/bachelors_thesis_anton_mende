\chapter{Ergebnisse}
\label{ch:ergebnisse}

\section{Analyse des LSS- und LSR-Formates}

- Keine Dokumentation zum Aufbau des XML vorhanden
- Lösung: Testdokumente erstellen und von da aus Format herausfinden
	- Erst simples Dokument mit Textfragen und zwei Fragegruppen

- Gruppen:
	- \el{groups} sammelt Metadaten über Fragegruppen, diese sind aber auch in \el{group\_l10ns}
	- \el{group\_l10ns} hat auch noch weitere Inhalte wie Name, Sprache

- \el{questions} Enthält Metadaten über Fragen, wie \el{qid}, \el{gid}, \el{title}, \el{type}.
- \el{subquestions} hat Unterfragen von Arrays, Matrizen etc. sind via \el{parent\_qid} and ein Element aus \el{question} gebunden
- \el{question\_l10ns} hat \el{qid} und \el{question}, \el{help}, \el{language}
	- Für jedes Element aus \el{question}, \el{subquestion} gibt es hier ein Element
- \el{question\_attributes} Enthält Informationen über eine Frage wie ein Prä-/Suffix, RegEx-Validations-Ausdrücke, Timings und Informationen zur Darstellung einer Frage (Textfeldbreite, Defauld-Antworten) 

- Antwortmöglichkeiten
	- werden entweder implizit durch den Typ ermittelt
	- oder (wenn custom) in \el{answers} (\el{qid}, \el{aid}, \el{code}) + \el{answer\_l10ns} (\el{answer}, \el{aid}, \el{language})

- \el{surveys} hat Metadaten über die Umfrage (keine für die Konvertierung relevanten)
- \el{surveyls\_languagesettings} enthält relevante Infos wie \el{surveyls\_id}, \el{surveyls\_title}, \el{surveyls\_description}
- \el{themes} und \el{themes\_inherited} enthalten Infos über die visuelle Darstellung der Umfrage in LimeSurvey

\subsection{Erstellung eines XML-Schemas}

\section{Mapping}

\subsection{Dummy Elemente in ODM}

- ODM kann mehr Beziehungen in Studien darstellen
- Es müssen Dummy-Elemente angelegt werden, weil LimeSurvey diese Elemente nicht hat, sie aber in ODM vorkommen müssen
- Betroffen: GlobalVariables, StudyEvent

\subsection{Umfrage-Eigenschaften}

- Eine Umfrage wird als ein \el{Form} dargestellt
- Titel, Beschreibung und ID der Umfrage werden zu Titel, Beschreibung und ID des \el{Forms}

\subsection{Fragegruppen}

- Aus jeder LS Gruppe wird eine ItemGroup in ODM
- Eigenschaften werden 1:1 abgebildet

\subsection{Fragen}

- Aus einer Frage in LS werden pot. mehrere ItemDef's in ODM (fragetypabhängig)
- Elemente aus \el{Questions} und \el{Subquestions} und \el{Question\_l10ns} werden genutzt

\subsection{Themes und Frageattribute}

- Werden nicht übernommen
	- ODM dient zum Austausch der Fragen/Antworten, nicht die Art und Weise, wie sie dargestellt werden

\section{Implementierung}

\subsection{Java}

- Bietet sehr mächtige Werkzeuge zur Bearbeitung von XML
	- Kann z.B. mit mehreren Attributen in einem Element umgehen
	- Unterstützt XSD-1.1
- Wird beim IMI bereits viel genutzt

\subsection{Eingabe}

- Archiv
	- muss entpackt werden

\subsection{XSD-Validierung}

- Nutze Xerces zusammen mit der erstellten XSD um eine Eingabedatei zu prüfen
- Prüfung ist nicht bindend
	- Auch invalide Datei wird weiter verarbeitet
	- Grund: Starke Versionsabhängigkeit
		 - Invalide lss-Datei kann eventuell trotzdem erfolgreich umgewandelt werden
	- Zur Information für Anwender: Potentiell problematisch/inkompatibel mit Konverter

\subsection{Parsing der LimeSurvey Struktur}

\subsection{Parsing der LimeSurvey Antworten}

\subsection{Ausgabe als ODM-Datei}

% Dieses Kapitel sollte die Ergebnisse beinhalten, die mit den Methoden aus \autoref{ch:methodik} erstellt wurden.

% \begin{table}[h]
% \caption{Beispieltabelle}
% \begin{center}
% 	\begin{tabular}{|c||c|c|}
% 		\hline
% 		Spalte1 & Spalte2 & Spalte3 \\ 
% 		\hline\hline
% 		   1    &    2    &    3    \\ 
% 		\hline
% 	\end{tabular}
% \end{center}
% \label{tbl:table}
% \end{table}
